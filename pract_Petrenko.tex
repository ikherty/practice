\documentclass[a4paper,12pt]{article}
\usepackage[backend=biber,sorting=none,style=gost-numeric,autolang=other]{biblatex} % библиография
\usepackage{mathtext} %русские буквы в формулах
\usepackage[T2A]{fontenc}
\usepackage[utf8]{inputenc}
\usepackage[english,russian]{babel}
\usepackage{amsmath}
\usepackage{fancyvrb}
\usepackage{formular}
\usepackage{setspace} % управление междустрочными интервалами
%поля документа
\usepackage[left=3cm,right=1cm,top=2cm,bottom=2cm]{geometry}

\usepackage{misccorr} % точки в конце номеров разделов, использовать перед пакетом ccaption!
\usepackage{ccaption} % изменения подписей к рисункам и табл.

\usepackage[nooneline]{caption} 
\captionsetup[table]{justification=raggedright} % заголовок таблицы выравнивается влево
\captionsetup[figure]{justification=centering,labelsep=endash} % заголовок рисунка - по центру
% отступ перед первым абзацем
\usepackage{indentfirst}
%вставка изображений
\usepackage{graphicx}
% счетчики
\usepackage{totcount}
% управление содержанием
\usepackage{tocloft}
\usepackage[english,russian]{babel}
% управление таблицами и рисунками
\usepackage{float}
% управление системами
\usepackage{cases}
\newcounter{mycitecount}                                %% Счётчик библиографии
\AtEveryBibitem{\stepcounter{mycitecount}}              %% Работает для biblatex

\usepackage[figure,      %
            table,       %
            mycitecount, xspace ]{totalcount}           %% Подсчёт общего количества объектов в документе

% окружение для листингов - с нумерацией строк слева
\DefineVerbatimEnvironment{MyCode}{Verbatim}{frame=lines,numbers=left,numberblanklines=false,framesep=5mm}

% автоматическая нумерация листингов
\newfloat{Program}{phb}{lop}
\floatname{Program}{Листинг}
\floatstyle{ruled}

\setcounter{secnumdepth}{3} % глубина нумерации до подразделов

%если нужны точки в оглавлении для разделов - раскомментируйте следующую команду
%\renewcommand{\cftsecleader}{\cftdotfill{\cftdotsep}}

\addto\captionsrussian{%
\renewcommand{\figurename}{Рисунок}%
\renewcommand{\tablename}{Таблица}%
}

% дефис в подписи к рисункам
\captiondelim{ -- } 

% Настройки для окружений с подчеркиваниями для подписей и пр.
\setFRMfontencoding{T2A}
\setFRMdfontencoding{T2A}
% thanks to A.Starikov
\setFRMfontfamily{cmr}
\setFRMdfontfamily{ptm}
\setFRMdfontsize{10pt}

% задает длину поля для подписи на титульной странице
\newFRMfield{xtitlesign}{32mm}

% поле для факультета или кафедры
\newFRMfield{fcath}{65mm}

%имя файла с библиографией в формате BibTex
\addbibresource{rbiblio.bib}

\begin{document}

% счетчики страниц, рисунков, таблиц
\regtotcounter{page}
\regtotcounter{figure}
\regtotcounter{table}

\renewcommand{\refname}{\centerline{СПИСОК ИСПОЛЬЗОВАННОЙ ЛИТЕРАТУРЫ}} 
\renewcommand{\contentsname}{\centerline{СОДЕРЖАНИЕ}} 
%\renewcommand{\refname}{Список источников}  % По умолчанию "Список литературы" (article)
%\renewcommand{\bibname}{Литература}  % По умолчанию "Литература" (book и report)

% титульная страница
\thispagestyle{empty}
\begin{center} \small
\textbf{МИНИСТЕРСТВО НАУКИ И ВЫСШЕГО ОБРАЗОВАНИЯ РОССИЙСКОЙ ФЕДЕРАЦИИ}\\
ФЕДЕРАЛЬНОЕ ГОСУДАРСТВЕННОЕ АВТОНОМНОЕ ОБРАЗОВАТЕЛЬНОЕ УЧРЕЖДЕНИЕ
ВЫСШЕГО  ОБРАЗОВАНИЯ\\
«Национальный исследовательский ядерный университет «МИФИ»\\
\textbf{Обнинский институт атомной энергетики} – \\
филиал федерального государственного автономного образовательного учреждения высшего\\
образования «Национальный исследовательский ядерный университет «МИФИ»\\
(ИАТЭ НИЯУ МИФИ)
\end{center}
%\vfill
\medskip

\begin{center}
\begin{tabular}{rl}
Отделение & \useFRMfield{fcath}[\large Интеллектуальные кибернетические системы] \\ 
Направление подготовки & \useFRMfield{fcath}[\large Информационные системы и технологии] \\ 
\end{tabular} 
\end{center}

\vfill

\large 

\begin{center}
	Производственная практика \\
	
	\medskip
	
	\textbf{\Large 
		Разработка наземной системы позиционирования и навигации микродрона с применением компьютерного зрения
	}
	
\end{center}

\vspace{1cm}

\begin{tabular*}{\textwidth}{lcr}
Студент группы ИС-Б17 & \useFRMfield{xtitlesign} & В. Ю. Петренко\\
& & \\
Руководитель & & \\
к.т.н., доцент каф. ИС & \useFRMfield{xtitlesign} & О. А. Мирзеабасов\\
& & \\
Научный консультант & & \\
Инженер ЦОНД & \useFRMfield{xtitlesign} & М. И. Турицын\\
& & \\
\end{tabular*}


\vfill
\large

\begin{center}
Обнинск, 2020 г
\end{center}

\onehalfspacing

\pagebreak

% реферат
\thispagestyle{empty}

\section*{\centering РЕФЕРАТ}
% объем не больше страницы, подобие аннотации
% возможно, кол-во источников придется вставлять вручную
\total{page} стр., \total{table} табл., \total{figure} рис. , \totalmycitecounts ист. 

% КЛЮЧЕВЫЕ СЛОВА В ВЕРХНЕМ РЕГИСТРЕ
БПЛА, КВАДРОКОПТЕР, РОБОТОТЕХНИКА, КОМПЬЮТЕРНОЕ ЗРЕНИЕ, АВТОНОМНЫЙ ПОЛЕТ, НАЗЕМНАЯ СТАНЦИЯ

Работа посвящена разработке экспериментального образца наземной станции и микродрона, а также выбору их протокола общения.
Суть работы заключается в модификации и доработке существующих робототехнических решений на основе БПЛА посредством выноса бортового компьютера в наземную станцию, благодаря чему станет возможно уменьшение размеров и стоимости квадрокоптера.
Наземная станция представляет собой совокупность компьютера и радиомодулей для получения и последующей обработки изображения с квадрокоптера. Результаты обработки отправляются по радио на борт квадрокоптера в виде управляющих сигналов.
Таким образом, квадрокоптер получает дополнительную систему координат без использования датчиков подобных GPS, а также возможность навигации с использованием компьютерного зрения.
% определяет свое место в пространстве 

Разработанный ПАК дает возможность создать образовательный робототехнический комплект на основе беспилотника. Вынося бортовой компьютер в наземную станцию, получаем такие преимущества, как безопасность, низкая стоимость, в том числе существенно удешевляется ремонт.
% простота эксплуатации
\pagebreak
\thispagestyle{empty}

\section*{\centering ОПРЕДЕЛЕНИЯ}

Квадрокоптер --- БПЛА мультироторного типа с четырьмя несущими винтами

UART -- universal asynchronous receiver/transmitter, аппаратный последовательный двусторонний интерфейс

\pagebreak

\section*{\centering ОБОЗНАЧЕНИЯ И СОКРАЩЕНИЯ}

НИР --- Научно-исследовательская работа

БПЛА --- Беспилотный летательный аппарат

ПАК --- Программно-аппаратный комплекс

MAVLink -- Micro Air Vehicle Link

ROS -- Robotic Operating System

FPV -- First Person View

ВМГ -- Винто-моторная группа

UART -- Universal asynchronous receiver/transmitter, аппаратный последовательный двусторонний интерфейс

%HITL -- hardware in the loop, программная симуляция аппарата
%SITL -- software in the loop, ПО для воссоздания поведения системы реального времени в симуляции
Бод-рейт -- скорость передачи данных через UART
\pagebreak

\tableofcontents

\pagebreak

\section*{\centering ВВЕДЕНИЕ}
\addcontentsline{toc}{section}{ВВЕДЕНИЕ}
Сфера беспилотных и робототехнических систем растет с неимоверной скоростью -- в 2020 году их количество увеличилось вдвое по сравнению с прошлым годом. На рынке труда спрос на специалистов данной области не удовлетворен и только растет. В связи с чем в программу основного общего образования включают курсы программирования и робототехники, закупаются образовательные комплекты на основе квадрокоптеров.

Цель данной научно -- исследовательской работы разработать аппаратную часть такого комплекта, а также описать протоколы взаимодействия наземной станции с квадрокоптером.
 % текст введения в файле intro.tex
\pagebreak
\pagebreak

\section{whatever}

\subsection{smth}
  % первая глава - в файле part1.tex
\pagebreak
% вторая часть

\section{smth}
\subsection{smth}

%//estimator lpe / ekf
%\url{https://dev.px4.io/v1.9.0/en/ros/offboard\_control.html}
%\url{https://dev.px4.io/v1.9.0/en/ros/external_position_estimation.html}
  % вторая глава - в файле part2.tex
\pagebreak

\section{smth}
  % третья глава - в файле part3.tex
\pagebreak

\section{smth}
  % четвертая глава - в файле part4.tex
\pagebreak

\section{smth}

  % пятая глава - в файле part5.tex
\pagebreak

\section*{\centering ЗАКЛЮЧЕНИЕ}
\addcontentsline{toc}{section}{ЗАКЛЮЧЕНИЕ}
В ходе проведения НИР были выполнены следующие задачи:
\begin{enumerate} 
	\item Изучение предметной области
	\item Разработка архитектуры микродрона
	\item Разработка архитектуры наземной станции
	\item Исследование протоколов общения дрона и наземной станции
	\item Создание экспериментального образца
\end{enumerate}
Спроектированный экспериментальный образец наземной станции и квадрокоптера позволяет разработать программную часть для автономных полетов квадрокоптера.

В теории, уже сейчас квадрокоптер готов к выполнению автономного полета. Однако из тестов видно, что задержка составляет порядка 200 мс, что критично для позиционирования квадрокоптера. Такая задержка обусловлена суммой задержек каждого устройства (камеры, видеопередатчика, видеоприемника, USB порта компьютера). Учитывая, что задержка также присутствует на устройстве приема-передачи телеметрии, обнаруживается критичная проблема у разрабатываемого комплекта. Уменьшение задержки -- одна из сложнейших задач, решению которых будет посвящен следующий этап исследования. Предполагаемые пути, которые позволят снизить задержки следующие:

--- подключение модуля телеметрии с более высоким бод -- рейтом,

--- модификация MAVLink протокола под наши нужды,

--- использование модификации MAVLink в виде fast-RTPS, включая уровень для преобразования сообщений uORB PX4.

Также огромной проблемой является качество и задержка видеоканала. Решение этой проблемы может быть в использовании цифровой системы передачи видеопотока, например, OpenHD.
\pagebreak

\addcontentsline{toc}{section}{СПИСОК ИСПОЛЬЗОВАННОЙ ЛИТЕРАТУРЫ}
% ВАЖНО: для корректного отображения в списке литературы ссылок на англ.языке в bibtex-описание источника следует добавить поле 
% langid = {english}
\printbibliography

\pagebreak

\end{document}          

