%
% Шаблон для ВКР бакалавра
%

\documentclass[a4paper,12pt]{article}
\usepackage[backend=biber,sorting=none,style=gost-numeric,autolang=other]{biblatex} % библиография
\usepackage{mathtext} %русские буквы в формулах
\usepackage[T2A]{fontenc}
\usepackage[utf8]{inputenc}
\usepackage[english,russian]{babel}
\usepackage{amsmath}
\usepackage{fancyvrb}
\usepackage{formular}
\usepackage{setspace} % управление междустрочными интервалами
%поля документа
\usepackage[left=3cm,right=1cm,top=2cm,bottom=2cm]{geometry}

\usepackage{misccorr} % точки в конце номеров разделов, использовать перед пакетом ccaption!
\usepackage{ccaption} % изменения подписей к рисункам и табл.

\usepackage[nooneline]{caption} 
\captionsetup[table]{justification=raggedright} % заголовок таблицы выравнивается влево
\captionsetup[figure]{justification=centering,labelsep=endash} % заголовок рисунка - по центру

% отступ перед первым абзацем
\usepackage{indentfirst}
%вставка изображений
\usepackage{graphicx}
% счетчики
\usepackage{totcount}
% управление содержанием
\usepackage{tocloft}
% управление таблицами и рисунками
\usepackage{float}

\newcounter{mycitecount}                                %% Счётчик библиографии
\AtEveryBibitem{\stepcounter{mycitecount}}              %% Работает для biblatex

\usepackage[figure,      %
            table,       %
            mycitecount, xspace ]{totalcount}           %% Подсчёт общего количества объектов в документе

% окружение для листингов - с нумерацией строк слева
\DefineVerbatimEnvironment{MyCode}{Verbatim}{frame=lines,numbers=left,numberblanklines=false,framesep=5mm}

% автоматическая нумерация листингов
\newfloat{Program}{phb}{lop}
\floatname{Program}{Листинг}
\floatstyle{ruled}

\setcounter{secnumdepth}{3} % глубина нумерации до подразделов

%если нужны точки в оглавлении для разделов - раскомментируйте следующую команду
%\renewcommand{\cftsecleader}{\cftdotfill{\cftdotsep}}

\addto\captionsrussian{%
\renewcommand{\figurename}{Рисунок}%
\renewcommand{\tablename}{Таблица}%
}

% дефис в подписи к рисункам
\captiondelim{ -- } 

% Настройки для окружений с подчеркиваниями для подписей и пр.
\setFRMfontencoding{T2A}
\setFRMdfontencoding{T2A}
% thanks to A.Starikov
\setFRMfontfamily{cmr}
\setFRMdfontfamily{ptm}
\setFRMdfontsize{10pt}

% задает длину поля для подписи на титульной странице
\newFRMfield{xtitlesign}{32mm}

% поле для факультета или кафедры
\newFRMfield{fcath}{65mm}

%имя файла с библиографией в формате BibTex
\addbibresource{rbiblio.bib}

\begin{document}

% счетчики страниц, рисунков, таблиц
\regtotcounter{page}
\regtotcounter{figure}
\regtotcounter{table}

\renewcommand{\refname}{\centerline{СПИСОК ИСПОЛЬЗОВАННОЙ ЛИТЕРАТУРЫ}} 
\renewcommand{\contentsname}{\centerline{СОДЕРЖАНИЕ}} 
%\renewcommand{\refname}{Список источников}  % По умолчанию "Список литературы" (article)
%\renewcommand{\bibname}{Литература}  % По умолчанию "Литература" (book и report)

% титульная страница
\thispagestyle{empty}
\begin{center} \small
\textbf{МИНИСТЕРСТВО НАУКИ И ВЫСШЕГО ОБРАЗОВАНИЯ\\ РОССИЙСКОЙ ФЕДЕРАЦИИ}\\
ФЕДЕРАЛЬНОЕ ГОСУДАРСТВЕННОЕ АВТОНОМНОЕ ОБРАЗОВАТЕЛЬНОЕ УЧРЕЖДЕНИЕ
ВЫСШЕГО  ОБРАЗОВАНИЯ\\
«Национальный исследовательский ядерный университет «МИФИ»\\
\textbf{Обнинский институт атомной энергетики} – \\
филиал федерального государственного автономного образовательного учреждения высшего\\
образования «Национальный исследовательский ядерный университет «МИФИ»\\
(ИАТЭ НИЯУ МИФИ)
\end{center}
%\vfill
\medskip

% Направление подготовки следует уточнять,
% магистры и бакалавры могут иметь разные наименования
\begin{center}
\begin{tabular}{rl}
Отделение & \useFRMfield{fcath}[\large Интеллектуальные кибернетические системы] \\ 
%Направление подготовки & \useFRMfield{fcath}[\large Информационные системы и технологии] \\ 
\end{tabular} 
\end{center}

\vfill

\large 

\begin{center}
\textbf{\Large Выпускная квалификационная работа --- } \\
\textbf{\Large бакалаврская работа}\\
	
	\medskip

{ \normalsize
по направлению подготовки  \textbf{09.03.02 Информационные  системы и технологии}\\

Направленность (профиль) \textbf{Информационные технологии}
}	
\vfill
\vfill
\medskip

\textbf{\Large 
		<<Полное наименование НИР в соответствии с заданием>>
	}
	
\end{center}

\vspace{1cm}

\begin{tabular*}{\textwidth}{p{78mm}p{33mm}p{64mm}}
	Выполнил:\\студент гр. ИС-М16 & \useFRMfield{xtitlesign} & Студе~Н.Т.\\
	& & \\
	Руководитель ВКР,\\доцент отделения ИКС, к.т.н. & \useFRMfield{xtitlesign} & Руководит~Е.Л. \\
	& & \\
	
	Нормоконтроль & \useFRMfield{xtitlesign} & Пичугина~И.А. \\
	& & \\
	% Если нужно добавить консультанта - раскомментируйте две строчки ниже
	%Консультант ВКР бакалавра\\организация, должность, звание  & \useFRMfield{xtitlesign} & К.О.Нсультант\\
	%& & \\
%	Рецензент\\к.ф.-м.н.,   & \useFRMfield{xtitlesign} & В.А. Чепурко\\
	
	& & \\
	Выпускная квалификационная \\ работа допущена к защите & \useFRMfield{xtitlesign} &  \\
	& & \\
	Руководитель\\ образовательной программы \\
	09.03.02 Информационные системы и технологии\\
	канд. тех. наук  & \useFRMfield{xtitlesign} &Мирзеабасов~О.А. \\
	
\end{tabular*}


\vfill
\large

\begin{center}
Обнинск, 2021 г
\end{center}

\onehalfspacing

\pagebreak

% реферат
\thispagestyle{empty}

\section*{\centering РЕФЕРАТ}

\total{page} стр., \total{table} табл., \total{figure} рис., \totalmycitecounts ист. 

КЛЮЧЕВЫЕ СЛОВА В ВЕРХНЕМ РЕГИСТРЕ

Работа посвящена --- кратко описать суть работы.

Разработанная программа дает возможность --- что именно.

\pagebreak
\thispagestyle{empty}

\section*{\centering ОПРЕДЕЛЕНИЯ}

Репозиторий --- место, где хранятся и поддерживаются версии каких-либо данных. 

\pagebreak

\section*{\centering ОБОЗНАЧЕНИЯ И СОКРАЩЕНИЯ}


НИР --- Научно-исследовательская работа

СПВ --- Система поддержки версий



\pagebreak



\tableofcontents
% если нужно добавить "Стр." над номерами страниц - раскомментируйте следующую команду
%\addtocontents{toc}{~\hfill\textbf{Стр.}\par}

\pagebreak

\section*{\centering ВВЕДЕНИЕ}
\addcontentsline{toc}{section}{ВВЕДЕНИЕ}
Сфера беспилотных и робототехнических систем растет с неимоверной скоростью -- в 2020 году их количество увеличилось вдвое по сравнению с прошлым годом. На рынке труда спрос на специалистов данной области не удовлетворен и только растет. В связи с чем в программу основного общего образования включают курсы программирования и робототехники, закупаются образовательные комплекты на основе квадрокоптеров.

Цель данной научно -- исследовательской работы разработать аппаратную часть такого комплекта, а также описать протоколы взаимодействия наземной станции с квадрокоптером.
 % текст введения в файле intro.tex
\pagebreak

%\input{Post_zad}
\pagebreak

\section{whatever}

\subsection{smth}
  % первая глава - в файле part1.tex
\pagebreak
% вторая часть

\section{smth}
\subsection{smth}

%//estimator lpe / ekf
%\url{https://dev.px4.io/v1.9.0/en/ros/offboard\_control.html}
%\url{https://dev.px4.io/v1.9.0/en/ros/external_position_estimation.html}
 % вторая глава - в файле part2.tex
\pagebreak

% если есть еще разделы - сохраните их в соответствующих файлах и раскомментируйте строки ниже, при необходимости добавьте еще
%
\section{smth}
  % третья глава - в файле part3.tex
%\pagebreak

%
\section{smth}
 % четвертая глава - в файле part4.tex
%\pagebreak

%
\section{smth}

  % пятая глава - в файле part5.tex
%\pagebreak

\section*{\centering ЗАКЛЮЧЕНИЕ}
\addcontentsline{toc}{section}{ЗАКЛЮЧЕНИЕ}

В ходе проделанной работы была проведена --- описать результаты бурной деятельности по выполнению НИР, разумно в виде списка выполненных задач.

Разработанная программа позволяет --- перечислить основные функциональные характеристики и особенности.

% оформление библиографии - вариант с БД
\pagebreak

\addcontentsline{toc}{section}{СПИСОК ИСПОЛЬЗОВАННОЙ ЛИТЕРАТУРЫ}
% ВАЖНО: для корректного отображения в списке литературы ссылок на англ.языке в bibtex-описание источника следует добавить поле 
% langid = {english}
\printbibliography

\pagebreak

\section*{ \centering Приложение А} 
\addcontentsline{toc}{section}{Приложение А}

\begin{center}
Листинг 1 -- 
\end{center}


\begin{MyCode}
///
\end{MyCode}

\end{document}          

